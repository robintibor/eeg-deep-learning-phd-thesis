%*******************************************************
% Abstract
%*******************************************************
%\renewcommand{\abstractname}{Abstract}
\pdfbookmark[1]{Abstract}{Abstract}
% \addcontentsline{toc}{chapter}{\tocEntry{Abstract}}
\begingroup
\let\clearpage\relax
\let\cleardoublepage\relax
\let\cleardoublepage\relax

\chapter*{Abstract}
Brain-signal decoding using machine learning can process larger amounts of signals and extract different information than humans can, with potential uses in medical diagnosis or brain-computer interfaces. In particular, brain-signal decoding from electroencephalographic (EEG) recordings is a promising area for machine learning due to the relative ease of acquiring large amounts of EEG recordings and the difficulty of interpreting them manually. Deep neural networks are a natural choice to train to decode EEG signals as they have been successful at a variety of natural-signal decoding tasks like object recognition from images or speech recognition from audio. However, prior to the work in this thesis, it was still unclear how well deep neural networks perform on EEG decoding compared to hand-engineered, feature-based approaches, and more research was needed to determine the optimal approaches for using deep learning to decode EEG. This thesis describes constructing and training deep neural networks for EEG decoding that perform at least as well as feature-based approaches and developing visualizations that suggest they both extract well-known physiologically plausible, as well as surprising features.

\vfill
\clearpage
\newpage

\begin{otherlanguage}{ngerman}
\pdfbookmark[1]{Zusammenfassung}{Zusammenfassung}
\chapter*{Zusammenfassung}
Die Dekodierung von Gehirnsignalen mit Hilfe von maschinellem Lernen kann größere Mengen von Signalen verarbeiten und andere Informationen extrahieren als der Mensch, was potenzielle Anwendungen in der medizinischen Diagnose oder bei Gehirn-Computer-Schnittstellen ermöglicht. Insbesondere die Dekodierung von Hirnsignalen aus elektroenzephalographischen (EEG) Aufzeichnungen ist ein vielversprechender Bereich für maschinelles Lernen, da es relativ einfach ist, größere Mengen von EEG-Aufzeichnungen aufzunehmen, und es schwierig ist, diese manuell zu interpretieren. Tiefe neuronale Netze sind eine natürliche Wahl als maschinelles Lernmodell für das Dekodieren von EEG-Signalen, da sie bei einer Vielzahl von Aufgaben zur Dekodierung natürlicher Signale, wie der Objekterkennung aus Bildern oder der Spracherkennung aus Audio, erfolgreich waren. Vor der in dieser Doktoarbeit vorgestellten Forschung war jedoch noch unklar, wie gut tiefe neuronale Netze bei der EEG-Dekodierung im Vergleich zu manuell entwickelten, merkmalsbasierten Ansätzen abschneiden, und es waren weitere Forschung erforderlich, um die optimalen Ansätze für die Verwendung von Deep Learning zur EEG-Dekodierung zu ermitteln. In dieser Arbeit wird beschrieben, die Konstruktion und das Training tiefe rneuronale Netze für die EEG-Dekodierung, die mindestens genauso gut wie merkmalsbasierte Ansätze funktionieren, sowie die Entwicklung von Visualisierungen, die nahelegen, dass sie sowohl bekannte physiologisch sinnvolle, als auch überraschende Merkmale extrahieren.
\end{otherlanguage}

\endgroup

\vfill
