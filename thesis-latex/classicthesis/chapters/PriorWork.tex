\documentclass[11pt]{article}

    \usepackage[breakable]{tcolorbox}
    \usepackage{parskip} % Stop auto-indenting (to mimic markdown behaviour)
    
    \usepackage{iftex}
    \ifPDFTeX
    	\usepackage[T1]{fontenc}
    	\usepackage{mathpazo}
    \else
    	\usepackage{fontspec}
    \fi

    % Basic figure setup, for now with no caption control since it's done
    % automatically by Pandoc (which extracts ![](path) syntax from Markdown).
    \usepackage{graphicx}
    % Maintain compatibility with old templates. Remove in nbconvert 6.0
    \let\Oldincludegraphics\includegraphics
    % Ensure that by default, figures have no caption (until we provide a
    % proper Figure object with a Caption API and a way to capture that
    % in the conversion process - todo).
    \usepackage{caption}
    \DeclareCaptionFormat{nocaption}{}
    \captionsetup{format=nocaption,aboveskip=0pt,belowskip=0pt}

    \usepackage{float}
    \floatplacement{figure}{H} % forces figures to be placed at the correct location
    \usepackage{xcolor} % Allow colors to be defined
    \usepackage{enumerate} % Needed for markdown enumerations to work
    \usepackage{geometry} % Used to adjust the document margins
    \usepackage{amsmath} % Equations
    \usepackage{amssymb} % Equations
    \usepackage{textcomp} % defines textquotesingle
    % Hack from http://tex.stackexchange.com/a/47451/13684:
    \AtBeginDocument{%
        \def\PYZsq{\textquotesingle}% Upright quotes in Pygmentized code
    }
    \usepackage{upquote} % Upright quotes for verbatim code
    \usepackage{eurosym} % defines \euro
    \usepackage[mathletters]{ucs} % Extended unicode (utf-8) support
    \usepackage{fancyvrb} % verbatim replacement that allows latex
    \usepackage{grffile} % extends the file name processing of package graphics 
                         % to support a larger range
    \makeatletter % fix for old versions of grffile with XeLaTeX
    \@ifpackagelater{grffile}{2019/11/01}
    {
      % Do nothing on new versions
    }
    {
      \def\Gread@@xetex#1{%
        \IfFileExists{"\Gin@base".bb}%
        {\Gread@eps{\Gin@base.bb}}%
        {\Gread@@xetex@aux#1}%
      }
    }
    \makeatother
    \usepackage[Export]{adjustbox} % Used to constrain images to a maximum size
    \adjustboxset{max size={0.9\linewidth}{0.9\paperheight}}

    % The hyperref package gives us a pdf with properly built
    % internal navigation ('pdf bookmarks' for the table of contents,
    % internal cross-reference links, web links for URLs, etc.)
    \usepackage{hyperref}
    % The default LaTeX title has an obnoxious amount of whitespace. By default,
    % titling removes some of it. It also provides customization options.
    \usepackage{titling}
    \usepackage{longtable} % longtable support required by pandoc >1.10
    \usepackage{booktabs}  % table support for pandoc > 1.12.2
    \usepackage[inline]{enumitem} % IRkernel/repr support (it uses the enumerate* environment)
    \usepackage[normalem]{ulem} % ulem is needed to support strikethroughs (\sout)
                                % normalem makes italics be italics, not underlines
    \usepackage{mathrsfs}
    

    
    % Colors for the hyperref package
    \definecolor{urlcolor}{rgb}{0,.145,.698}
    \definecolor{linkcolor}{rgb}{.71,0.21,0.01}
    \definecolor{citecolor}{rgb}{.12,.54,.11}

    % ANSI colors
    \definecolor{ansi-black}{HTML}{3E424D}
    \definecolor{ansi-black-intense}{HTML}{282C36}
    \definecolor{ansi-red}{HTML}{E75C58}
    \definecolor{ansi-red-intense}{HTML}{B22B31}
    \definecolor{ansi-green}{HTML}{00A250}
    \definecolor{ansi-green-intense}{HTML}{007427}
    \definecolor{ansi-yellow}{HTML}{DDB62B}
    \definecolor{ansi-yellow-intense}{HTML}{B27D12}
    \definecolor{ansi-blue}{HTML}{208FFB}
    \definecolor{ansi-blue-intense}{HTML}{0065CA}
    \definecolor{ansi-magenta}{HTML}{D160C4}
    \definecolor{ansi-magenta-intense}{HTML}{A03196}
    \definecolor{ansi-cyan}{HTML}{60C6C8}
    \definecolor{ansi-cyan-intense}{HTML}{258F8F}
    \definecolor{ansi-white}{HTML}{C5C1B4}
    \definecolor{ansi-white-intense}{HTML}{A1A6B2}
    \definecolor{ansi-default-inverse-fg}{HTML}{FFFFFF}
    \definecolor{ansi-default-inverse-bg}{HTML}{000000}

    % common color for the border for error outputs.
    \definecolor{outerrorbackground}{HTML}{FFDFDF}

    % commands and environments needed by pandoc snippets
    % extracted from the output of `pandoc -s`
    \providecommand{\tightlist}{%
      \setlength{\itemsep}{0pt}\setlength{\parskip}{0pt}}
    \DefineVerbatimEnvironment{Highlighting}{Verbatim}{commandchars=\\\{\}}
    % Add ',fontsize=\small' for more characters per line
    \newenvironment{Shaded}{}{}
    \newcommand{\KeywordTok}[1]{\textcolor[rgb]{0.00,0.44,0.13}{\textbf{{#1}}}}
    \newcommand{\DataTypeTok}[1]{\textcolor[rgb]{0.56,0.13,0.00}{{#1}}}
    \newcommand{\DecValTok}[1]{\textcolor[rgb]{0.25,0.63,0.44}{{#1}}}
    \newcommand{\BaseNTok}[1]{\textcolor[rgb]{0.25,0.63,0.44}{{#1}}}
    \newcommand{\FloatTok}[1]{\textcolor[rgb]{0.25,0.63,0.44}{{#1}}}
    \newcommand{\CharTok}[1]{\textcolor[rgb]{0.25,0.44,0.63}{{#1}}}
    \newcommand{\StringTok}[1]{\textcolor[rgb]{0.25,0.44,0.63}{{#1}}}
    \newcommand{\CommentTok}[1]{\textcolor[rgb]{0.38,0.63,0.69}{\textit{{#1}}}}
    \newcommand{\OtherTok}[1]{\textcolor[rgb]{0.00,0.44,0.13}{{#1}}}
    \newcommand{\AlertTok}[1]{\textcolor[rgb]{1.00,0.00,0.00}{\textbf{{#1}}}}
    \newcommand{\FunctionTok}[1]{\textcolor[rgb]{0.02,0.16,0.49}{{#1}}}
    \newcommand{\RegionMarkerTok}[1]{{#1}}
    \newcommand{\ErrorTok}[1]{\textcolor[rgb]{1.00,0.00,0.00}{\textbf{{#1}}}}
    \newcommand{\NormalTok}[1]{{#1}}
    
    % Additional commands for more recent versions of Pandoc
    \newcommand{\ConstantTok}[1]{\textcolor[rgb]{0.53,0.00,0.00}{{#1}}}
    \newcommand{\SpecialCharTok}[1]{\textcolor[rgb]{0.25,0.44,0.63}{{#1}}}
    \newcommand{\VerbatimStringTok}[1]{\textcolor[rgb]{0.25,0.44,0.63}{{#1}}}
    \newcommand{\SpecialStringTok}[1]{\textcolor[rgb]{0.73,0.40,0.53}{{#1}}}
    \newcommand{\ImportTok}[1]{{#1}}
    \newcommand{\DocumentationTok}[1]{\textcolor[rgb]{0.73,0.13,0.13}{\textit{{#1}}}}
    \newcommand{\AnnotationTok}[1]{\textcolor[rgb]{0.38,0.63,0.69}{\textbf{\textit{{#1}}}}}
    \newcommand{\CommentVarTok}[1]{\textcolor[rgb]{0.38,0.63,0.69}{\textbf{\textit{{#1}}}}}
    \newcommand{\VariableTok}[1]{\textcolor[rgb]{0.10,0.09,0.49}{{#1}}}
    \newcommand{\ControlFlowTok}[1]{\textcolor[rgb]{0.00,0.44,0.13}{\textbf{{#1}}}}
    \newcommand{\OperatorTok}[1]{\textcolor[rgb]{0.40,0.40,0.40}{{#1}}}
    \newcommand{\BuiltInTok}[1]{{#1}}
    \newcommand{\ExtensionTok}[1]{{#1}}
    \newcommand{\PreprocessorTok}[1]{\textcolor[rgb]{0.74,0.48,0.00}{{#1}}}
    \newcommand{\AttributeTok}[1]{\textcolor[rgb]{0.49,0.56,0.16}{{#1}}}
    \newcommand{\InformationTok}[1]{\textcolor[rgb]{0.38,0.63,0.69}{\textbf{\textit{{#1}}}}}
    \newcommand{\WarningTok}[1]{\textcolor[rgb]{0.38,0.63,0.69}{\textbf{\textit{{#1}}}}}
    
    
    % Define a nice break command that doesn't care if a line doesn't already
    % exist.
    \def\br{\hspace*{\fill} \\* }
    % Math Jax compatibility definitions
    \def\gt{>}
    \def\lt{<}
    \let\Oldtex\TeX
    \let\Oldlatex\LaTeX
    \renewcommand{\TeX}{\textrm{\Oldtex}}
    \renewcommand{\LaTeX}{\textrm{\Oldlatex}}
    % Document parameters
    % Document title
    \title{PriorWork}
    
    
    
    
    
% Pygments definitions
\makeatletter
\def\PY@reset{\let\PY@it=\relax \let\PY@bf=\relax%
    \let\PY@ul=\relax \let\PY@tc=\relax%
    \let\PY@bc=\relax \let\PY@ff=\relax}
\def\PY@tok#1{\csname PY@tok@#1\endcsname}
\def\PY@toks#1+{\ifx\relax#1\empty\else%
    \PY@tok{#1}\expandafter\PY@toks\fi}
\def\PY@do#1{\PY@bc{\PY@tc{\PY@ul{%
    \PY@it{\PY@bf{\PY@ff{#1}}}}}}}
\def\PY#1#2{\PY@reset\PY@toks#1+\relax+\PY@do{#2}}

\@namedef{PY@tok@w}{\def\PY@tc##1{\textcolor[rgb]{0.73,0.73,0.73}{##1}}}
\@namedef{PY@tok@c}{\let\PY@it=\textit\def\PY@tc##1{\textcolor[rgb]{0.24,0.48,0.48}{##1}}}
\@namedef{PY@tok@cp}{\def\PY@tc##1{\textcolor[rgb]{0.61,0.40,0.00}{##1}}}
\@namedef{PY@tok@k}{\let\PY@bf=\textbf\def\PY@tc##1{\textcolor[rgb]{0.00,0.50,0.00}{##1}}}
\@namedef{PY@tok@kp}{\def\PY@tc##1{\textcolor[rgb]{0.00,0.50,0.00}{##1}}}
\@namedef{PY@tok@kt}{\def\PY@tc##1{\textcolor[rgb]{0.69,0.00,0.25}{##1}}}
\@namedef{PY@tok@o}{\def\PY@tc##1{\textcolor[rgb]{0.40,0.40,0.40}{##1}}}
\@namedef{PY@tok@ow}{\let\PY@bf=\textbf\def\PY@tc##1{\textcolor[rgb]{0.67,0.13,1.00}{##1}}}
\@namedef{PY@tok@nb}{\def\PY@tc##1{\textcolor[rgb]{0.00,0.50,0.00}{##1}}}
\@namedef{PY@tok@nf}{\def\PY@tc##1{\textcolor[rgb]{0.00,0.00,1.00}{##1}}}
\@namedef{PY@tok@nc}{\let\PY@bf=\textbf\def\PY@tc##1{\textcolor[rgb]{0.00,0.00,1.00}{##1}}}
\@namedef{PY@tok@nn}{\let\PY@bf=\textbf\def\PY@tc##1{\textcolor[rgb]{0.00,0.00,1.00}{##1}}}
\@namedef{PY@tok@ne}{\let\PY@bf=\textbf\def\PY@tc##1{\textcolor[rgb]{0.80,0.25,0.22}{##1}}}
\@namedef{PY@tok@nv}{\def\PY@tc##1{\textcolor[rgb]{0.10,0.09,0.49}{##1}}}
\@namedef{PY@tok@no}{\def\PY@tc##1{\textcolor[rgb]{0.53,0.00,0.00}{##1}}}
\@namedef{PY@tok@nl}{\def\PY@tc##1{\textcolor[rgb]{0.46,0.46,0.00}{##1}}}
\@namedef{PY@tok@ni}{\let\PY@bf=\textbf\def\PY@tc##1{\textcolor[rgb]{0.44,0.44,0.44}{##1}}}
\@namedef{PY@tok@na}{\def\PY@tc##1{\textcolor[rgb]{0.41,0.47,0.13}{##1}}}
\@namedef{PY@tok@nt}{\let\PY@bf=\textbf\def\PY@tc##1{\textcolor[rgb]{0.00,0.50,0.00}{##1}}}
\@namedef{PY@tok@nd}{\def\PY@tc##1{\textcolor[rgb]{0.67,0.13,1.00}{##1}}}
\@namedef{PY@tok@s}{\def\PY@tc##1{\textcolor[rgb]{0.73,0.13,0.13}{##1}}}
\@namedef{PY@tok@sd}{\let\PY@it=\textit\def\PY@tc##1{\textcolor[rgb]{0.73,0.13,0.13}{##1}}}
\@namedef{PY@tok@si}{\let\PY@bf=\textbf\def\PY@tc##1{\textcolor[rgb]{0.64,0.35,0.47}{##1}}}
\@namedef{PY@tok@se}{\let\PY@bf=\textbf\def\PY@tc##1{\textcolor[rgb]{0.67,0.36,0.12}{##1}}}
\@namedef{PY@tok@sr}{\def\PY@tc##1{\textcolor[rgb]{0.64,0.35,0.47}{##1}}}
\@namedef{PY@tok@ss}{\def\PY@tc##1{\textcolor[rgb]{0.10,0.09,0.49}{##1}}}
\@namedef{PY@tok@sx}{\def\PY@tc##1{\textcolor[rgb]{0.00,0.50,0.00}{##1}}}
\@namedef{PY@tok@m}{\def\PY@tc##1{\textcolor[rgb]{0.40,0.40,0.40}{##1}}}
\@namedef{PY@tok@gh}{\let\PY@bf=\textbf\def\PY@tc##1{\textcolor[rgb]{0.00,0.00,0.50}{##1}}}
\@namedef{PY@tok@gu}{\let\PY@bf=\textbf\def\PY@tc##1{\textcolor[rgb]{0.50,0.00,0.50}{##1}}}
\@namedef{PY@tok@gd}{\def\PY@tc##1{\textcolor[rgb]{0.63,0.00,0.00}{##1}}}
\@namedef{PY@tok@gi}{\def\PY@tc##1{\textcolor[rgb]{0.00,0.52,0.00}{##1}}}
\@namedef{PY@tok@gr}{\def\PY@tc##1{\textcolor[rgb]{0.89,0.00,0.00}{##1}}}
\@namedef{PY@tok@ge}{\let\PY@it=\textit}
\@namedef{PY@tok@gs}{\let\PY@bf=\textbf}
\@namedef{PY@tok@gp}{\let\PY@bf=\textbf\def\PY@tc##1{\textcolor[rgb]{0.00,0.00,0.50}{##1}}}
\@namedef{PY@tok@go}{\def\PY@tc##1{\textcolor[rgb]{0.44,0.44,0.44}{##1}}}
\@namedef{PY@tok@gt}{\def\PY@tc##1{\textcolor[rgb]{0.00,0.27,0.87}{##1}}}
\@namedef{PY@tok@err}{\def\PY@bc##1{{\setlength{\fboxsep}{\string -\fboxrule}\fcolorbox[rgb]{1.00,0.00,0.00}{1,1,1}{\strut ##1}}}}
\@namedef{PY@tok@kc}{\let\PY@bf=\textbf\def\PY@tc##1{\textcolor[rgb]{0.00,0.50,0.00}{##1}}}
\@namedef{PY@tok@kd}{\let\PY@bf=\textbf\def\PY@tc##1{\textcolor[rgb]{0.00,0.50,0.00}{##1}}}
\@namedef{PY@tok@kn}{\let\PY@bf=\textbf\def\PY@tc##1{\textcolor[rgb]{0.00,0.50,0.00}{##1}}}
\@namedef{PY@tok@kr}{\let\PY@bf=\textbf\def\PY@tc##1{\textcolor[rgb]{0.00,0.50,0.00}{##1}}}
\@namedef{PY@tok@bp}{\def\PY@tc##1{\textcolor[rgb]{0.00,0.50,0.00}{##1}}}
\@namedef{PY@tok@fm}{\def\PY@tc##1{\textcolor[rgb]{0.00,0.00,1.00}{##1}}}
\@namedef{PY@tok@vc}{\def\PY@tc##1{\textcolor[rgb]{0.10,0.09,0.49}{##1}}}
\@namedef{PY@tok@vg}{\def\PY@tc##1{\textcolor[rgb]{0.10,0.09,0.49}{##1}}}
\@namedef{PY@tok@vi}{\def\PY@tc##1{\textcolor[rgb]{0.10,0.09,0.49}{##1}}}
\@namedef{PY@tok@vm}{\def\PY@tc##1{\textcolor[rgb]{0.10,0.09,0.49}{##1}}}
\@namedef{PY@tok@sa}{\def\PY@tc##1{\textcolor[rgb]{0.73,0.13,0.13}{##1}}}
\@namedef{PY@tok@sb}{\def\PY@tc##1{\textcolor[rgb]{0.73,0.13,0.13}{##1}}}
\@namedef{PY@tok@sc}{\def\PY@tc##1{\textcolor[rgb]{0.73,0.13,0.13}{##1}}}
\@namedef{PY@tok@dl}{\def\PY@tc##1{\textcolor[rgb]{0.73,0.13,0.13}{##1}}}
\@namedef{PY@tok@s2}{\def\PY@tc##1{\textcolor[rgb]{0.73,0.13,0.13}{##1}}}
\@namedef{PY@tok@sh}{\def\PY@tc##1{\textcolor[rgb]{0.73,0.13,0.13}{##1}}}
\@namedef{PY@tok@s1}{\def\PY@tc##1{\textcolor[rgb]{0.73,0.13,0.13}{##1}}}
\@namedef{PY@tok@mb}{\def\PY@tc##1{\textcolor[rgb]{0.40,0.40,0.40}{##1}}}
\@namedef{PY@tok@mf}{\def\PY@tc##1{\textcolor[rgb]{0.40,0.40,0.40}{##1}}}
\@namedef{PY@tok@mh}{\def\PY@tc##1{\textcolor[rgb]{0.40,0.40,0.40}{##1}}}
\@namedef{PY@tok@mi}{\def\PY@tc##1{\textcolor[rgb]{0.40,0.40,0.40}{##1}}}
\@namedef{PY@tok@il}{\def\PY@tc##1{\textcolor[rgb]{0.40,0.40,0.40}{##1}}}
\@namedef{PY@tok@mo}{\def\PY@tc##1{\textcolor[rgb]{0.40,0.40,0.40}{##1}}}
\@namedef{PY@tok@ch}{\let\PY@it=\textit\def\PY@tc##1{\textcolor[rgb]{0.24,0.48,0.48}{##1}}}
\@namedef{PY@tok@cm}{\let\PY@it=\textit\def\PY@tc##1{\textcolor[rgb]{0.24,0.48,0.48}{##1}}}
\@namedef{PY@tok@cpf}{\let\PY@it=\textit\def\PY@tc##1{\textcolor[rgb]{0.24,0.48,0.48}{##1}}}
\@namedef{PY@tok@c1}{\let\PY@it=\textit\def\PY@tc##1{\textcolor[rgb]{0.24,0.48,0.48}{##1}}}
\@namedef{PY@tok@cs}{\let\PY@it=\textit\def\PY@tc##1{\textcolor[rgb]{0.24,0.48,0.48}{##1}}}

\def\PYZbs{\char`\\}
\def\PYZus{\char`\_}
\def\PYZob{\char`\{}
\def\PYZcb{\char`\}}
\def\PYZca{\char`\^}
\def\PYZam{\char`\&}
\def\PYZlt{\char`\<}
\def\PYZgt{\char`\>}
\def\PYZsh{\char`\#}
\def\PYZpc{\char`\%}
\def\PYZdl{\char`\$}
\def\PYZhy{\char`\-}
\def\PYZsq{\char`\'}
\def\PYZdq{\char`\"}
\def\PYZti{\char`\~}
% for compatibility with earlier versions
\def\PYZat{@}
\def\PYZlb{[}
\def\PYZrb{]}
\makeatother


    % For linebreaks inside Verbatim environment from package fancyvrb. 
    \makeatletter
        \newbox\Wrappedcontinuationbox 
        \newbox\Wrappedvisiblespacebox 
        \newcommand*\Wrappedvisiblespace {\textcolor{red}{\textvisiblespace}} 
        \newcommand*\Wrappedcontinuationsymbol {\textcolor{red}{\llap{\tiny$\m@th\hookrightarrow$}}} 
        \newcommand*\Wrappedcontinuationindent {3ex } 
        \newcommand*\Wrappedafterbreak {\kern\Wrappedcontinuationindent\copy\Wrappedcontinuationbox} 
        % Take advantage of the already applied Pygments mark-up to insert 
        % potential linebreaks for TeX processing. 
        %        {, <, #, %, $, ' and ": go to next line. 
        %        _, }, ^, &, >, - and ~: stay at end of broken line. 
        % Use of \textquotesingle for straight quote. 
        \newcommand*\Wrappedbreaksatspecials {% 
            \def\PYGZus{\discretionary{\char`\_}{\Wrappedafterbreak}{\char`\_}}% 
            \def\PYGZob{\discretionary{}{\Wrappedafterbreak\char`\{}{\char`\{}}% 
            \def\PYGZcb{\discretionary{\char`\}}{\Wrappedafterbreak}{\char`\}}}% 
            \def\PYGZca{\discretionary{\char`\^}{\Wrappedafterbreak}{\char`\^}}% 
            \def\PYGZam{\discretionary{\char`\&}{\Wrappedafterbreak}{\char`\&}}% 
            \def\PYGZlt{\discretionary{}{\Wrappedafterbreak\char`\<}{\char`\<}}% 
            \def\PYGZgt{\discretionary{\char`\>}{\Wrappedafterbreak}{\char`\>}}% 
            \def\PYGZsh{\discretionary{}{\Wrappedafterbreak\char`\#}{\char`\#}}% 
            \def\PYGZpc{\discretionary{}{\Wrappedafterbreak\char`\%}{\char`\%}}% 
            \def\PYGZdl{\discretionary{}{\Wrappedafterbreak\char`\$}{\char`\$}}% 
            \def\PYGZhy{\discretionary{\char`\-}{\Wrappedafterbreak}{\char`\-}}% 
            \def\PYGZsq{\discretionary{}{\Wrappedafterbreak\textquotesingle}{\textquotesingle}}% 
            \def\PYGZdq{\discretionary{}{\Wrappedafterbreak\char`\"}{\char`\"}}% 
            \def\PYGZti{\discretionary{\char`\~}{\Wrappedafterbreak}{\char`\~}}% 
        } 
        % Some characters . , ; ? ! / are not pygmentized. 
        % This macro makes them "active" and they will insert potential linebreaks 
        \newcommand*\Wrappedbreaksatpunct {% 
            \lccode`\~`\.\lowercase{\def~}{\discretionary{\hbox{\char`\.}}{\Wrappedafterbreak}{\hbox{\char`\.}}}% 
            \lccode`\~`\,\lowercase{\def~}{\discretionary{\hbox{\char`\,}}{\Wrappedafterbreak}{\hbox{\char`\,}}}% 
            \lccode`\~`\;\lowercase{\def~}{\discretionary{\hbox{\char`\;}}{\Wrappedafterbreak}{\hbox{\char`\;}}}% 
            \lccode`\~`\:\lowercase{\def~}{\discretionary{\hbox{\char`\:}}{\Wrappedafterbreak}{\hbox{\char`\:}}}% 
            \lccode`\~`\?\lowercase{\def~}{\discretionary{\hbox{\char`\?}}{\Wrappedafterbreak}{\hbox{\char`\?}}}% 
            \lccode`\~`\!\lowercase{\def~}{\discretionary{\hbox{\char`\!}}{\Wrappedafterbreak}{\hbox{\char`\!}}}% 
            \lccode`\~`\/\lowercase{\def~}{\discretionary{\hbox{\char`\/}}{\Wrappedafterbreak}{\hbox{\char`\/}}}% 
            \catcode`\.\active
            \catcode`\,\active 
            \catcode`\;\active
            \catcode`\:\active
            \catcode`\?\active
            \catcode`\!\active
            \catcode`\/\active 
            \lccode`\~`\~ 	
        }
    \makeatother

    \let\OriginalVerbatim=\Verbatim
    \makeatletter
    \renewcommand{\Verbatim}[1][1]{%
        %\parskip\z@skip
        \sbox\Wrappedcontinuationbox {\Wrappedcontinuationsymbol}%
        \sbox\Wrappedvisiblespacebox {\FV@SetupFont\Wrappedvisiblespace}%
        \def\FancyVerbFormatLine ##1{\hsize\linewidth
            \vtop{\raggedright\hyphenpenalty\z@\exhyphenpenalty\z@
                \doublehyphendemerits\z@\finalhyphendemerits\z@
                \strut ##1\strut}%
        }%
        % If the linebreak is at a space, the latter will be displayed as visible
        % space at end of first line, and a continuation symbol starts next line.
        % Stretch/shrink are however usually zero for typewriter font.
        \def\FV@Space {%
            \nobreak\hskip\z@ plus\fontdimen3\font minus\fontdimen4\font
            \discretionary{\copy\Wrappedvisiblespacebox}{\Wrappedafterbreak}
            {\kern\fontdimen2\font}%
        }%
        
        % Allow breaks at special characters using \PYG... macros.
        \Wrappedbreaksatspecials
        % Breaks at punctuation characters . , ; ? ! and / need catcode=\active 	
        \OriginalVerbatim[#1,codes*=\Wrappedbreaksatpunct]%
    }
    \makeatother

    % Exact colors from NB
    \definecolor{incolor}{HTML}{303F9F}
    \definecolor{outcolor}{HTML}{D84315}
    \definecolor{cellborder}{HTML}{CFCFCF}
    \definecolor{cellbackground}{HTML}{F7F7F7}
    
    % prompt
    \makeatletter
    \newcommand{\boxspacing}{\kern\kvtcb@left@rule\kern\kvtcb@boxsep}
    \makeatother
    \newcommand{\prompt}[4]{
        {\ttfamily\llap{{\color{#2}[#3]:\hspace{3pt}#4}}\vspace{-\baselineskip}}
    }
    

    
    % Prevent overflowing lines due to hard-to-break entities
    \sloppy 
    % Setup hyperref package
    \hypersetup{
      breaklinks=true,  % so long urls are correctly broken across lines
      colorlinks=true,
      urlcolor=urlcolor,
      linkcolor=linkcolor,
      citecolor=citecolor,
      }
    % Slightly bigger margins than the latex defaults
    
    \geometry{verbose,tmargin=1in,bmargin=1in,lmargin=1in,rmargin=1in}
    
    

\begin{document}
    
    \maketitle
    
    

    
    (prior-work)= \# Prior Work

    \texttt{\{admonition\}\ \ Prior\ to\ our\ work,\ research\ on\ deep-learning-based\ EEG\ decoding\ was\ limited\ *\ Few\ studies\ compared\ to\ published\ feature-based\ decoding\ results\ *\ Most\ EEG\ DL\ architectures\ had\ only\ 1-3\ convolutional\ layers\ and\ included\ fully-connected\ layers\ with\ many\ parameters\ *\ Most\ work\ only\ considered\ very\ restricted\ frequency\ ranges\ *\ Most\ studies\ only\ compared\ few\ design\ choices\ and\ training\ strategies}

    Prior to 2017, when the first work presented in this thesis was
published, there was only limited literature on EEG decoding with deep
learning. In this chapter, I outline what decoding problems, input
representations, network architectures, hyperparameter choices and
visualizations were evaluated in prior work. This is based on the
literature research that we presented in
\citet{schirrmeisterdeephbm2017}.

    \hypertarget{decoding-problems-and-baselines}{%
\section{Decoding Problems and
Baselines}\label{decoding-problems-and-baselines}}

    ```\{table\} \textbf{Decoding problems in deep-learning EEG decoding
studies prior to our work.} Studies with published baseline compared
their decoding results to an external baseline result published by other
authors. :name: prior-work-tasks-table

\begin{longtable}[]{@{}lll@{}}
\toprule
Decoding problem & Number of studies & With published baseline \\
\midrule
\endhead
Imagined or Executed Movement & 6 & 2 \\
Oddball/P300 & 5 & 1 \\
Epilepsy-related & 4 & 2 \\
Music Rhythm & 2 & 0 \\
Memory Performance/Cognitive Load & 2 & 0 \\
Driver Performance & 1 & 0 \\
\bottomrule
\end{longtable}

```

    The most widely studied decoding problems were movement-related decoding
problems such as decoding which body part (hand, feet etc.) a person is
moving or imagining to move (see
\Cref{prior-work-tasks-table}). From the 19 studies we
identified at the time, only 5 compared their decoding results to an
external published baseline result, limiting the insights about
deep-learning EEG decoding performance. We therefore decided to compare
deep-learning EEG decoding to a strong feature-based baseline (see
\Cref{fbscp-and-filterbank-net}) on widely researched
movement-related decoding tasks.

    \hypertarget{input-domains-and-frequency-ranges}{%
\section{Input Domains and Frequency
Ranges}\label{input-domains-and-frequency-ranges}}

    \begin{tcolorbox}[breakable, size=fbox, boxrule=1pt, pad at break*=1mm,colback=cellbackground, colframe=cellborder]
\prompt{In}{incolor}{1}{\boxspacing}
\begin{Verbatim}[commandchars=\\\{\}]
\PY{k+kn}{import} \PY{n+nn}{matplotlib}
\PY{k+kn}{import} \PY{n+nn}{matplotlib}\PY{n+nn}{.}\PY{n+nn}{pyplot} \PY{k}{as} \PY{n+nn}{plt}
\PY{k+kn}{from} \PY{n+nn}{matplotlib} \PY{k+kn}{import} \PY{n}{cm}
\PY{k+kn}{import} \PY{n+nn}{seaborn}
\PY{k+kn}{import} \PY{n+nn}{numpy} \PY{k}{as} \PY{n+nn}{np}
\PY{k+kn}{import} \PY{n+nn}{re}
\PY{k+kn}{from} \PY{n+nn}{myst\PYZus{}nb} \PY{k+kn}{import} \PY{n}{glue}
\PY{n}{seaborn}\PY{o}{.}\PY{n}{set\PYZus{}palette}\PY{p}{(}\PY{l+s+s1}{\PYZsq{}}\PY{l+s+s1}{colorblind}\PY{l+s+s1}{\PYZsq{}}\PY{p}{)}
\PY{n}{seaborn}\PY{o}{.}\PY{n}{set\PYZus{}style}\PY{p}{(}\PY{l+s+s1}{\PYZsq{}}\PY{l+s+s1}{darkgrid}\PY{l+s+s1}{\PYZsq{}}\PY{p}{)}
\PY{k+kn}{import} \PY{n+nn}{re}

\PY{o}{\PYZpc{}}\PY{k}{matplotlib} inline
\PY{o}{\PYZpc{}}\PY{k}{config} InlineBackend.figure\PYZus{}format = \PYZsq{}png\PYZsq{}
\PY{c+c1}{\PYZsh{}matplotlib.rcParams[\PYZsq{}figure.figsize\PYZsq{}] = (12.0, 1.0)}
\PY{n}{matplotlib}\PY{o}{.}\PY{n}{rcParams}\PY{p}{[}\PY{l+s+s1}{\PYZsq{}}\PY{l+s+s1}{font.size}\PY{l+s+s1}{\PYZsq{}}\PY{p}{]} \PY{o}{=} \PY{l+m+mi}{14}
\PY{n}{a} \PY{o}{=} \PY{n}{np}\PY{o}{.}\PY{n}{array}\PY{p}{(}\PY{p}{[}\PY{l+s+s1}{\PYZsq{}}\PY{l+s+s1}{Time,  8–30 Hz }\PY{l+s+s1}{\PYZsq{}}\PY{p}{,} \PY{l+s+s1}{\PYZsq{}}\PY{l+s+s1}{Time, 0.1–40 Hz }\PY{l+s+s1}{\PYZsq{}}\PY{p}{,} \PY{l+s+s1}{\PYZsq{}}\PY{l+s+s1}{Time, 0.05–15 Hz }\PY{l+s+s1}{\PYZsq{}}\PY{p}{,}
       \PY{l+s+s1}{\PYZsq{}}\PY{l+s+s1}{Time, 0.3–20 Hz }\PY{l+s+s1}{\PYZsq{}}\PY{p}{,} \PY{l+s+s1}{\PYZsq{}}\PY{l+s+s1}{Frequency, 6–30 Hz }\PY{l+s+s1}{\PYZsq{}}\PY{p}{,} \PY{l+s+s1}{\PYZsq{}}\PY{l+s+s1}{ Frequency, 0–200 Hz }\PY{l+s+s1}{\PYZsq{}}\PY{p}{,}
       \PY{l+s+s1}{\PYZsq{}}\PY{l+s+s1}{Time,  1–50 Hz }\PY{l+s+s1}{\PYZsq{}}\PY{p}{,} \PY{l+s+s1}{\PYZsq{}}\PY{l+s+s1}{ Time,  0–100 HZ }\PY{l+s+s1}{\PYZsq{}}\PY{p}{,}
       \PY{l+s+s1}{\PYZsq{}}\PY{l+s+s1}{Frequency, mean amplitude for 0–7 Hz, 7–14 Hz, 14–49 Hz }\PY{l+s+s1}{\PYZsq{}}\PY{p}{,}
       \PY{l+s+s1}{\PYZsq{}}\PY{l+s+s1}{Time, 0.5–50 Hz }\PY{l+s+s1}{\PYZsq{}}\PY{p}{,} \PY{l+s+s1}{\PYZsq{}}\PY{l+s+s1}{Time,  0–128 Hz }\PY{l+s+s1}{\PYZsq{}}\PY{p}{,}
       \PY{l+s+s1}{\PYZsq{}}\PY{l+s+s1}{ Frequency, mean power for 4–7 Hz, 8–13 Hz, 13–30 Hz }\PY{l+s+s1}{\PYZsq{}}\PY{p}{,}
       \PY{l+s+s1}{\PYZsq{}}\PY{l+s+s1}{Time, 0.5–30Hz }\PY{l+s+s1}{\PYZsq{}}\PY{p}{,} \PY{l+s+s1}{\PYZsq{}}\PY{l+s+s1}{Time, 0.1–50 Hz }\PY{l+s+s1}{\PYZsq{}}\PY{p}{,}
       \PY{l+s+s1}{\PYZsq{}}\PY{l+s+s1}{Frequency, 4–40 Hz, using FBCSP }\PY{l+s+s1}{\PYZsq{}}\PY{p}{,}
       \PY{l+s+s1}{\PYZsq{}}\PY{l+s+s1}{ Time and frequency evaluated, 0\PYZhy{}200 Hz }\PY{l+s+s1}{\PYZsq{}}\PY{p}{,} \PY{l+s+s1}{\PYZsq{}}\PY{l+s+s1}{Frequency, 8–30 Hz }\PY{l+s+s1}{\PYZsq{}}\PY{p}{,}
       \PY{l+s+s1}{\PYZsq{}}\PY{l+s+s1}{Time, 0.15–200 Hz }\PY{l+s+s1}{\PYZsq{}}\PY{p}{,} \PY{l+s+s1}{\PYZsq{}}\PY{l+s+s1}{ Time, 0.1\PYZhy{}20 Hz }\PY{l+s+s1}{\PYZsq{}}\PY{p}{]}\PY{p}{)}
\PY{n}{domain\PYZus{}strings} \PY{o}{=} \PY{p}{[}\PY{n}{s}\PY{o}{.}\PY{n}{split}\PY{p}{(}\PY{l+s+s1}{\PYZsq{}}\PY{l+s+s1}{,}\PY{l+s+s1}{\PYZsq{}}\PY{p}{)}\PY{p}{[}\PY{l+m+mi}{0}\PY{p}{]} \PY{k}{for} \PY{n}{s} \PY{o+ow}{in} \PY{n}{a}\PY{p}{]}
\PY{n}{start\PYZus{}fs} \PY{o}{=} \PY{p}{[}\PY{n+nb}{float}\PY{p}{(}\PY{n}{re}\PY{o}{.}\PY{n}{sub}\PY{p}{(}\PY{l+s+sa}{r}\PY{l+s+s1}{\PYZsq{}}\PY{l+s+s1}{[a\PYZhy{}z ]+}\PY{l+s+s1}{\PYZsq{}}\PY{p}{,}\PY{l+s+sa}{r}\PY{l+s+s1}{\PYZsq{}}\PY{l+s+s1}{\PYZsq{}}\PY{p}{,} \PY{n}{re}\PY{o}{.}\PY{n}{split}\PY{p}{(}\PY{l+s+sa}{r}\PY{l+s+s1}{\PYZsq{}}\PY{l+s+s1}{[–\PYZhy{}–\PYZhy{}]}\PY{l+s+s1}{\PYZsq{}}\PY{p}{,}\PY{l+s+s2}{\PYZdq{}}\PY{l+s+s2}{ }\PY{l+s+s2}{\PYZdq{}}\PY{o}{.}\PY{n}{join}\PY{p}{(}\PY{n}{s}\PY{o}{.}\PY{n}{split}\PY{p}{(}\PY{l+s+s1}{\PYZsq{}}\PY{l+s+s1}{,}\PY{l+s+s1}{\PYZsq{}}\PY{p}{)}\PY{p}{[}\PY{l+m+mi}{1}\PY{p}{:}\PY{p}{]}\PY{p}{)}\PY{p}{)}\PY{p}{[}\PY{l+m+mi}{0}\PY{p}{]}\PY{p}{)}\PY{p}{)} \PY{k}{for} \PY{n}{s} \PY{o+ow}{in} \PY{n}{a}\PY{p}{]}
\PY{n}{end\PYZus{}fs} \PY{o}{=} \PY{p}{[}\PY{n+nb}{float}\PY{p}{(}\PY{n}{re}\PY{o}{.}\PY{n}{sub}\PY{p}{(}\PY{l+s+sa}{r}\PY{l+s+s1}{\PYZsq{}}\PY{l+s+s1}{[a\PYZhy{}z HZFBCSP]+}\PY{l+s+s1}{\PYZsq{}}\PY{p}{,}\PY{l+s+sa}{r}\PY{l+s+s1}{\PYZsq{}}\PY{l+s+s1}{\PYZsq{}}\PY{p}{,} \PY{n}{re}\PY{o}{.}\PY{n}{split}\PY{p}{(}\PY{l+s+sa}{r}\PY{l+s+s1}{\PYZsq{}}\PY{l+s+s1}{[–\PYZhy{}–\PYZhy{}]}\PY{l+s+s1}{\PYZsq{}}\PY{p}{,}\PY{l+s+s2}{\PYZdq{}}\PY{l+s+s2}{ }\PY{l+s+s2}{\PYZdq{}}\PY{o}{.}\PY{n}{join}\PY{p}{(}\PY{n}{s}\PY{o}{.}\PY{n}{split}\PY{p}{(}\PY{l+s+s1}{\PYZsq{}}\PY{l+s+s1}{,}\PY{l+s+s1}{\PYZsq{}}\PY{p}{)}\PY{p}{[}\PY{l+m+mi}{1}\PY{p}{:}\PY{p}{]}\PY{p}{)}\PY{p}{)}\PY{p}{[}\PY{l+m+mi}{1}\PY{p}{]}\PY{p}{)}\PY{p}{)} \PY{k}{for} \PY{n}{s} \PY{o+ow}{in} \PY{n}{a}\PY{p}{]}
\PY{n}{domain\PYZus{}strings} \PY{o}{=} \PY{n}{np}\PY{o}{.}\PY{n}{array}\PY{p}{(}\PY{n}{domain\PYZus{}strings}\PY{p}{)}
\PY{n}{start\PYZus{}fs} \PY{o}{=} \PY{n}{np}\PY{o}{.}\PY{n}{array}\PY{p}{(}\PY{n}{start\PYZus{}fs}\PY{p}{)}
\PY{n}{end\PYZus{}fs} \PY{o}{=} \PY{n}{np}\PY{o}{.}\PY{n}{array}\PY{p}{(}\PY{n}{end\PYZus{}fs}\PY{p}{)}

\PY{n}{freq\PYZus{}mask} \PY{o}{=} \PY{n}{np}\PY{o}{.}\PY{n}{array}\PY{p}{(}\PY{p}{[}\PY{l+s+s1}{\PYZsq{}}\PY{l+s+s1}{freq}\PY{l+s+s1}{\PYZsq{}} \PY{o+ow}{in} \PY{n}{s}\PY{o}{.}\PY{n}{lower}\PY{p}{(}\PY{p}{)} \PY{k}{for} \PY{n}{s} \PY{o+ow}{in} \PY{n}{domain\PYZus{}strings}\PY{p}{]}\PY{p}{)}
\PY{n}{time\PYZus{}mask} \PY{o}{=} \PY{n}{np}\PY{o}{.}\PY{n}{array}\PY{p}{(}\PY{p}{[}\PY{l+s+s1}{\PYZsq{}}\PY{l+s+s1}{time}\PY{l+s+s1}{\PYZsq{}} \PY{o+ow}{in} \PY{n}{s}\PY{o}{.}\PY{n}{lower}\PY{p}{(}\PY{p}{)} \PY{k}{for} \PY{n}{s} \PY{o+ow}{in} \PY{n}{domain\PYZus{}strings}\PY{p}{]}\PY{p}{)}

\PY{n}{fig} \PY{o}{=} \PY{n}{plt}\PY{o}{.}\PY{n}{figure}\PY{p}{(}\PY{n}{figsize}\PY{o}{=}\PY{p}{(}\PY{l+m+mi}{8}\PY{p}{,}\PY{l+m+mi}{4}\PY{p}{)}\PY{p}{)}
\PY{n}{rng} \PY{o}{=} \PY{n}{np}\PY{o}{.}\PY{n}{random}\PY{o}{.}\PY{n}{RandomState}\PY{p}{(}\PY{l+m+mi}{98349384}\PY{p}{)}
\PY{n}{color} \PY{o}{=} \PY{n}{seaborn}\PY{o}{.}\PY{n}{color\PYZus{}palette}\PY{p}{(}\PY{p}{)}\PY{p}{[}\PY{l+m+mi}{0}\PY{p}{]}
\PY{n}{i\PYZus{}sort} \PY{o}{=} \PY{n}{np}\PY{o}{.}\PY{n}{flatnonzero}\PY{p}{(}\PY{n}{time\PYZus{}mask}\PY{p}{)}\PY{p}{[}\PY{n}{np}\PY{o}{.}\PY{n}{argsort}\PY{p}{(}\PY{n}{end\PYZus{}fs}\PY{p}{[}\PY{n}{time\PYZus{}mask}\PY{p}{]}\PY{p}{)}\PY{p}{]}
\PY{k}{for} \PY{n}{i}\PY{p}{,} \PY{p}{(}\PY{n}{d}\PY{p}{,}\PY{n}{s}\PY{p}{,}\PY{n}{e}\PY{p}{)} \PY{o+ow}{in} \PY{n+nb}{enumerate}\PY{p}{(}\PY{n+nb}{zip}\PY{p}{(}
        \PY{n}{domain\PYZus{}strings}\PY{p}{[}\PY{n}{i\PYZus{}sort}\PY{p}{]}\PY{p}{,} \PY{n}{start\PYZus{}fs}\PY{p}{[}\PY{n}{i\PYZus{}sort}\PY{p}{]}\PY{p}{,} \PY{n}{end\PYZus{}fs}\PY{p}{[}\PY{n}{i\PYZus{}sort}\PY{p}{]}\PY{p}{)}\PY{p}{)}\PY{p}{:}
    \PY{n}{offset} \PY{o}{=} \PY{l+m+mf}{0.6}\PY{o}{*}\PY{n}{i}\PY{o}{/}\PY{n+nb}{len}\PY{p}{(}\PY{n}{i\PYZus{}sort}\PY{p}{)} \PY{o}{\PYZhy{}} \PY{l+m+mf}{0.3}
    \PY{n}{plt}\PY{o}{.}\PY{n}{plot}\PY{p}{(}\PY{p}{[}\PY{n}{offset}\PY{p}{,}\PY{n}{offset}\PY{p}{]} \PY{p}{,} \PY{p}{[}\PY{n}{s}\PY{p}{,} \PY{n}{e}\PY{p}{]}\PY{p}{,} \PY{n}{marker}\PY{o}{=}\PY{l+s+s1}{\PYZsq{}}\PY{l+s+s1}{o}\PY{l+s+s1}{\PYZsq{}}\PY{p}{,} \PY{n}{alpha}\PY{o}{=}\PY{l+m+mi}{1}\PY{p}{,} \PY{n}{color}\PY{o}{=}\PY{n}{color}\PY{p}{,} \PY{n}{ls}\PY{o}{=}\PY{l+s+s1}{\PYZsq{}}\PY{l+s+s1}{\PYZhy{}}\PY{l+s+s1}{\PYZsq{}}\PY{p}{)}
\PY{n}{i\PYZus{}sort} \PY{o}{=} \PY{n}{np}\PY{o}{.}\PY{n}{flatnonzero}\PY{p}{(}\PY{n}{freq\PYZus{}mask}\PY{p}{)}\PY{p}{[}\PY{n}{np}\PY{o}{.}\PY{n}{argsort}\PY{p}{(}\PY{n}{end\PYZus{}fs}\PY{p}{[}\PY{n}{freq\PYZus{}mask}\PY{p}{]}\PY{p}{)}\PY{p}{]}
\PY{k}{for} \PY{n}{i}\PY{p}{,} \PY{p}{(}\PY{n}{d}\PY{p}{,}\PY{n}{s}\PY{p}{,}\PY{n}{e}\PY{p}{)} \PY{o+ow}{in} \PY{n+nb}{enumerate}\PY{p}{(}\PY{n+nb}{zip}\PY{p}{(}
        \PY{n}{domain\PYZus{}strings}\PY{p}{[}\PY{n}{i\PYZus{}sort}\PY{p}{]}\PY{p}{,} \PY{n}{start\PYZus{}fs}\PY{p}{[}\PY{n}{i\PYZus{}sort}\PY{p}{]}\PY{p}{,} \PY{n}{end\PYZus{}fs}\PY{p}{[}\PY{n}{i\PYZus{}sort}\PY{p}{]}\PY{p}{)}\PY{p}{)}\PY{p}{:}
    \PY{n}{offset} \PY{o}{=} \PY{l+m+mf}{0.6}\PY{o}{*}\PY{n}{i}\PY{o}{/}\PY{n+nb}{len}\PY{p}{(}\PY{n}{i\PYZus{}sort}\PY{p}{)} \PY{o}{+} \PY{l+m+mf}{0.7}
    \PY{n}{plt}\PY{o}{.}\PY{n}{plot}\PY{p}{(}\PY{p}{[}\PY{n}{offset}\PY{p}{,}\PY{n}{offset}\PY{p}{]} \PY{p}{,} \PY{p}{[}\PY{n}{s}\PY{p}{,} \PY{n}{e}\PY{p}{]}\PY{p}{,} \PY{n}{marker}\PY{o}{=}\PY{l+s+s1}{\PYZsq{}}\PY{l+s+s1}{o}\PY{l+s+s1}{\PYZsq{}}\PY{p}{,} \PY{n}{alpha}\PY{o}{=}\PY{l+m+mi}{1}\PY{p}{,} \PY{n}{color}\PY{o}{=}\PY{n}{color}\PY{p}{,} \PY{n}{ls}\PY{o}{=}\PY{l+s+s1}{\PYZsq{}}\PY{l+s+s1}{\PYZhy{}}\PY{l+s+s1}{\PYZsq{}}\PY{p}{)}

\PY{n}{plt}\PY{o}{.}\PY{n}{xlim}\PY{p}{(}\PY{o}{\PYZhy{}}\PY{l+m+mf}{0.5}\PY{p}{,}\PY{l+m+mf}{1.5}\PY{p}{)}
\PY{n}{plt}\PY{o}{.}\PY{n}{xlabel}\PY{p}{(}\PY{l+s+s2}{\PYZdq{}}\PY{l+s+s2}{Input domain}\PY{l+s+s2}{\PYZdq{}}\PY{p}{)}
\PY{n}{plt}\PY{o}{.}\PY{n}{ylabel}\PY{p}{(}\PY{l+s+s2}{\PYZdq{}}\PY{l+s+s2}{Frequency [Hz]}\PY{l+s+s2}{\PYZdq{}}\PY{p}{)}
\PY{n}{plt}\PY{o}{.}\PY{n}{xticks}\PY{p}{(}\PY{p}{[}\PY{l+m+mi}{0}\PY{p}{,}\PY{l+m+mi}{1}\PY{p}{]}\PY{p}{,} \PY{p}{[}\PY{l+s+s2}{\PYZdq{}}\PY{l+s+s2}{Time}\PY{l+s+s2}{\PYZdq{}}\PY{p}{,} \PY{l+s+s2}{\PYZdq{}}\PY{l+s+s2}{Frequency}\PY{l+s+s2}{\PYZdq{}}\PY{p}{]}\PY{p}{,} \PY{n}{rotation}\PY{o}{=}\PY{l+m+mi}{45}\PY{p}{)}
\PY{n}{plt}\PY{o}{.}\PY{n}{title}\PY{p}{(}\PY{l+s+s2}{\PYZdq{}}\PY{l+s+s2}{Input domains and frequency ranges in prior work}\PY{l+s+s2}{\PYZdq{}}\PY{p}{,} \PY{n}{y}\PY{o}{=}\PY{l+m+mf}{1.05}\PY{p}{)}
\PY{n}{plt}\PY{o}{.}\PY{n}{yticks}\PY{p}{(}\PY{p}{[}\PY{l+m+mi}{0}\PY{p}{,}\PY{l+m+mi}{25}\PY{p}{,}\PY{l+m+mi}{50}\PY{p}{,}\PY{l+m+mi}{75}\PY{p}{,}\PY{l+m+mi}{100}\PY{p}{,}\PY{l+m+mi}{150}\PY{p}{,}\PY{l+m+mi}{200}\PY{p}{]}\PY{p}{)}
\PY{n}{glue}\PY{p}{(}\PY{l+s+s1}{\PYZsq{}}\PY{l+s+s1}{input\PYZus{}domain\PYZus{}fig}\PY{l+s+s1}{\PYZsq{}}\PY{p}{,} \PY{n}{fig}\PY{p}{)}
\PY{n}{plt}\PY{o}{.}\PY{n}{close}\PY{p}{(}\PY{n}{fig}\PY{p}{)}
\PY{k+kc}{None}
\end{Verbatim}
\end{tcolorbox}

    \begin{center}
    \adjustimage{max size={0.9\linewidth}{0.9\paperheight}}{PriorWork_files/PriorWork_7_0.png}
    \end{center}
    { \hspace*{\fill} \\}
    
    ```\{glue:figure\} input\_domain\_fig

\textbf{Input domains and frequency ranges in prior work.} Grey lines
represent frequency ranges of individual studies. Note that many studies
only include frequencies below 50 Hz, some use very restricted ranges
(alpha/beta band). ```

    Deep networks can either decode directly from the time-domain EEG or
process the data in the frequency domain, for example after a Fourier
transformation. 12 of the prior studies used time-domain inputs, 6 used
frequency-domain inputs and one used both. We decided to work directly
in the time domain, as the deep networks should in principle be able to
learn how to extract any needed spectral information from the
time-domain input.

Most prior studies that were working in the time domain only used
frequencies below 50 Hz. We were interested in how well deep networks
can also extract lesser-used higher-frequency components of the EEG
signal. For that, we used a sampling rate of 250 Hz, which means we were
able to analyze frequencies up to the Nyquist frequency of 125 Hz. As a
suitable dataset where high-frequency information may help decoding, we
included our high-gamma dataset in our study, since it was recorded
specifically to allow extraction of higher-frequency (\textgreater50 Hz)
information from scalp EEG \citep{schirrmeisterdeephbm2017}.

    \hypertarget{network-architectures}{%
\section{Network Architectures}\label{network-architectures}}

    \begin{tcolorbox}[breakable, size=fbox, boxrule=1pt, pad at break*=1mm,colback=cellbackground, colframe=cellborder]
\prompt{In}{incolor}{2}{\boxspacing}
\begin{Verbatim}[commandchars=\\\{\}]
\PY{n}{ls} \PY{o}{=} \PY{n}{np}\PY{o}{.}\PY{n}{array}\PY{p}{(}\PY{p}{[}\PY{l+s+s1}{\PYZsq{}}\PY{l+s+s1}{ 2/2 }\PY{l+s+s1}{\PYZsq{}}\PY{p}{,} \PY{l+s+s1}{\PYZsq{}}\PY{l+s+s1}{ 3/1 }\PY{l+s+s1}{\PYZsq{}}\PY{p}{,} \PY{l+s+s1}{\PYZsq{}}\PY{l+s+s1}{ 2/2 }\PY{l+s+s1}{\PYZsq{}}\PY{p}{,} \PY{l+s+s1}{\PYZsq{}}\PY{l+s+s1}{ 3/2 }\PY{l+s+s1}{\PYZsq{}}\PY{p}{,} \PY{l+s+s1}{\PYZsq{}}\PY{l+s+s1}{ 1/1 }\PY{l+s+s1}{\PYZsq{}}\PY{p}{,} \PY{l+s+s1}{\PYZsq{}}\PY{l+s+s1}{ 1/2 }\PY{l+s+s1}{\PYZsq{}}\PY{p}{,} \PY{l+s+s1}{\PYZsq{}}\PY{l+s+s1}{ 1/3 }\PY{l+s+s1}{\PYZsq{}}\PY{p}{,}
       \PY{l+s+s1}{\PYZsq{}}\PY{l+s+s1}{ 1–2/2 }\PY{l+s+s1}{\PYZsq{}}\PY{p}{,} \PY{l+s+s1}{\PYZsq{}}\PY{l+s+s1}{ 3/1 (+ LSTM as postprocessor) }\PY{l+s+s1}{\PYZsq{}}\PY{p}{,} \PY{l+s+s1}{\PYZsq{}}\PY{l+s+s1}{ 4/3 }\PY{l+s+s1}{\PYZsq{}}\PY{p}{,} \PY{l+s+s1}{\PYZsq{}}\PY{l+s+s1}{ 1\PYZhy{}3/1\PYZhy{}3 }\PY{l+s+s1}{\PYZsq{}}\PY{p}{,}
       \PY{l+s+s1}{\PYZsq{}}\PY{l+s+s1}{ 3–7/2 (+ LSTM or other temporal post\PYZhy{}processing (see design choices)) }\PY{l+s+s1}{\PYZsq{}}\PY{p}{,}
       \PY{l+s+s1}{\PYZsq{}}\PY{l+s+s1}{ 2/1 }\PY{l+s+s1}{\PYZsq{}}\PY{p}{,} \PY{l+s+s1}{\PYZsq{}}\PY{l+s+s1}{ 3/3 (Spatio\PYZhy{}temporal regularization) }\PY{l+s+s1}{\PYZsq{}}\PY{p}{,}
       \PY{l+s+s1}{\PYZsq{}}\PY{l+s+s1}{ 2/2 (Final fully connected layer uses concatenated output by convolutionaland fully connected layers) }\PY{l+s+s1}{\PYZsq{}}\PY{p}{,}
       \PY{l+s+s1}{\PYZsq{}}\PY{l+s+s1}{ 1\PYZhy{}2/1 }\PY{l+s+s1}{\PYZsq{}}\PY{p}{,}
       \PY{l+s+s1}{\PYZsq{}}\PY{l+s+s1}{2/0 (Convolutional deep belief network, separately trained RBF\PYZhy{}SVM classifier) }\PY{l+s+s1}{\PYZsq{}}\PY{p}{,}
       \PY{l+s+s1}{\PYZsq{}}\PY{l+s+s1}{ 3/1 (Convolutional layers trained as convolutional stacked autoencoder with target prior) }\PY{l+s+s1}{\PYZsq{}}\PY{p}{,}
       \PY{l+s+s1}{\PYZsq{}}\PY{l+s+s1}{ 2/2 }\PY{l+s+s1}{\PYZsq{}}\PY{p}{]}\PY{p}{)}

\PY{n}{conv\PYZus{}ls} \PY{o}{=} \PY{p}{[}\PY{n}{l}\PY{o}{.}\PY{n}{split}\PY{p}{(}\PY{l+s+s1}{\PYZsq{}}\PY{l+s+s1}{/}\PY{l+s+s1}{\PYZsq{}}\PY{p}{)}\PY{p}{[}\PY{l+m+mi}{0}\PY{p}{]} \PY{k}{for} \PY{n}{l} \PY{o+ow}{in} \PY{n}{ls}\PY{p}{]}
\PY{n}{low\PYZus{}conv\PYZus{}ls} \PY{o}{=} \PY{p}{[}\PY{n+nb}{int}\PY{p}{(}\PY{n}{re}\PY{o}{.}\PY{n}{split}\PY{p}{(}\PY{l+s+sa}{r}\PY{l+s+s1}{\PYZsq{}}\PY{l+s+s1}{[–\PYZhy{}]}\PY{l+s+s1}{\PYZsq{}}\PY{p}{,} \PY{n}{c}\PY{p}{)}\PY{p}{[}\PY{l+m+mi}{0}\PY{p}{]}\PY{p}{)}\PY{k}{for} \PY{n}{c} \PY{o+ow}{in} \PY{n}{conv\PYZus{}ls}\PY{p}{]}
\PY{n}{high\PYZus{}conv\PYZus{}ls} \PY{o}{=} \PY{p}{[}\PY{n+nb}{int}\PY{p}{(}\PY{n}{re}\PY{o}{.}\PY{n}{split}\PY{p}{(}\PY{l+s+sa}{r}\PY{l+s+s1}{\PYZsq{}}\PY{l+s+s1}{[–\PYZhy{}]}\PY{l+s+s1}{\PYZsq{}}\PY{p}{,} \PY{n}{c}\PY{p}{)}\PY{p}{[}\PY{o}{\PYZhy{}}\PY{l+m+mi}{1}\PY{p}{]}\PY{p}{)}\PY{k}{for} \PY{n}{c} \PY{o+ow}{in} \PY{n}{conv\PYZus{}ls}\PY{p}{]}
\PY{n}{dense\PYZus{}ls} \PY{o}{=} \PY{p}{[}\PY{n}{l}\PY{o}{.}\PY{n}{split}\PY{p}{(}\PY{l+s+s1}{\PYZsq{}}\PY{l+s+s1}{/}\PY{l+s+s1}{\PYZsq{}}\PY{p}{)}\PY{p}{[}\PY{l+m+mi}{1}\PY{p}{]} \PY{k}{for} \PY{n}{l} \PY{o+ow}{in} \PY{n}{ls}\PY{p}{]}
\PY{n}{low\PYZus{}dense\PYZus{}ls} \PY{o}{=} \PY{p}{[}\PY{n+nb}{int}\PY{p}{(}\PY{n}{re}\PY{o}{.}\PY{n}{split}\PY{p}{(}\PY{l+s+sa}{r}\PY{l+s+s1}{\PYZsq{}}\PY{l+s+s1}{[–\PYZhy{}]}\PY{l+s+s1}{\PYZsq{}}\PY{p}{,} \PY{n}{c}\PY{p}{[}\PY{p}{:}\PY{l+m+mi}{8}\PY{p}{]}\PY{p}{)}\PY{p}{[}\PY{l+m+mi}{0}\PY{p}{]}\PY{p}{[}\PY{p}{:}\PY{l+m+mi}{2}\PY{p}{]}\PY{p}{)}\PY{k}{for} \PY{n}{c} \PY{o+ow}{in} \PY{n}{dense\PYZus{}ls}\PY{p}{]}
\PY{n}{high\PYZus{}dense\PYZus{}ls} \PY{o}{=} \PY{p}{[}\PY{n+nb}{int}\PY{p}{(}\PY{n}{re}\PY{o}{.}\PY{n}{split}\PY{p}{(}\PY{l+s+sa}{r}\PY{l+s+s1}{\PYZsq{}}\PY{l+s+s1}{[–\PYZhy{}]}\PY{l+s+s1}{\PYZsq{}}\PY{p}{,} \PY{n}{c}\PY{p}{[}\PY{p}{:}\PY{l+m+mi}{8}\PY{p}{]}\PY{p}{)}\PY{p}{[}\PY{o}{\PYZhy{}}\PY{l+m+mi}{1}\PY{p}{]}\PY{p}{[}\PY{p}{:}\PY{l+m+mi}{2}\PY{p}{]}\PY{p}{)}\PY{k}{for} \PY{n}{c} \PY{o+ow}{in} \PY{n}{dense\PYZus{}ls}\PY{p}{]}

\PY{n}{all\PYZus{}conv\PYZus{}ls} \PY{o}{=} \PY{n}{np}\PY{o}{.}\PY{n}{concatenate}\PY{p}{(}\PY{p}{[}\PY{n}{np}\PY{o}{.}\PY{n}{arange}\PY{p}{(}\PY{n}{low\PYZus{}c}\PY{p}{,} \PY{n}{high\PYZus{}c}\PY{o}{+}\PY{l+m+mi}{1}\PY{p}{)} \PY{k}{for} \PY{n}{low\PYZus{}c}\PY{p}{,} \PY{n}{high\PYZus{}c} \PY{o+ow}{in} \PY{n+nb}{zip}\PY{p}{(}\PY{n}{low\PYZus{}conv\PYZus{}ls}\PY{p}{,} \PY{n}{high\PYZus{}conv\PYZus{}ls}\PY{p}{)}\PY{p}{]}\PY{p}{)}
\PY{n}{all\PYZus{}dense\PYZus{}ls} \PY{o}{=} \PY{n}{np}\PY{o}{.}\PY{n}{concatenate}\PY{p}{(}\PY{p}{[}\PY{n}{np}\PY{o}{.}\PY{n}{arange}\PY{p}{(}\PY{n}{low\PYZus{}c}\PY{p}{,} \PY{n}{high\PYZus{}c}\PY{o}{+}\PY{l+m+mi}{1}\PY{p}{)} \PY{k}{for} \PY{n}{low\PYZus{}c}\PY{p}{,} \PY{n}{high\PYZus{}c} \PY{o+ow}{in} \PY{n+nb}{zip}\PY{p}{(}\PY{n}{low\PYZus{}dense\PYZus{}ls}\PY{p}{,} \PY{n}{high\PYZus{}dense\PYZus{}ls}\PY{p}{)}\PY{p}{]}\PY{p}{)}
\PY{n}{bincount\PYZus{}conv} \PY{o}{=} \PY{n}{np}\PY{o}{.}\PY{n}{bincount}\PY{p}{(}\PY{n}{all\PYZus{}conv\PYZus{}ls}\PY{p}{)}
\PY{n}{bincount\PYZus{}dense} \PY{o}{=} \PY{n}{np}\PY{o}{.}\PY{n}{bincount}\PY{p}{(}\PY{n}{all\PYZus{}dense\PYZus{}ls}\PY{p}{)}
\PY{n}{rng} \PY{o}{=} \PY{n}{np}\PY{o}{.}\PY{n}{random}\PY{o}{.}\PY{n}{RandomState}\PY{p}{(}\PY{l+m+mi}{98349384}\PY{p}{)}
\PY{n}{color} \PY{o}{=} \PY{n}{seaborn}\PY{o}{.}\PY{n}{color\PYZus{}palette}\PY{p}{(}\PY{p}{)}\PY{p}{[}\PY{l+m+mi}{0}\PY{p}{]}
\PY{n}{fig} \PY{o}{=} \PY{n}{plt}\PY{o}{.}\PY{n}{figure}\PY{p}{(}\PY{n}{figsize}\PY{o}{=}\PY{p}{(}\PY{l+m+mi}{8}\PY{p}{,}\PY{l+m+mi}{4}\PY{p}{)}\PY{p}{)}
\PY{k}{for} \PY{n}{low\PYZus{}c}\PY{p}{,} \PY{n}{high\PYZus{}c} \PY{o+ow}{in} \PY{n+nb}{zip}\PY{p}{(}\PY{n}{low\PYZus{}conv\PYZus{}ls}\PY{p}{,} \PY{n}{high\PYZus{}conv\PYZus{}ls}\PY{p}{)}\PY{p}{:}
    \PY{n}{offset} \PY{o}{=} \PY{n}{rng}\PY{o}{.}\PY{n}{randn}\PY{p}{(}\PY{l+m+mi}{1}\PY{p}{)} \PY{o}{*} \PY{l+m+mf}{0.1}
    \PY{n}{tried\PYZus{}cs} \PY{o}{=} \PY{n}{np}\PY{o}{.}\PY{n}{arange}\PY{p}{(}\PY{n}{low\PYZus{}c}\PY{p}{,} \PY{n}{high\PYZus{}c}\PY{o}{+}\PY{l+m+mi}{1}\PY{p}{)}
    \PY{n}{plt}\PY{o}{.}\PY{n}{plot}\PY{p}{(}\PY{p}{[}\PY{n}{offset}\PY{p}{,}\PY{p}{]} \PY{o}{*} \PY{n+nb}{len}\PY{p}{(}\PY{n}{tried\PYZus{}cs}\PY{p}{)}\PY{p}{,} \PY{n}{tried\PYZus{}cs}\PY{p}{,} \PY{n}{marker}\PY{o}{=}\PY{l+s+s1}{\PYZsq{}}\PY{l+s+s1}{o}\PY{l+s+s1}{\PYZsq{}}\PY{p}{,} \PY{n}{alpha}\PY{o}{=}\PY{l+m+mf}{0.5}\PY{p}{,} \PY{n}{color}\PY{o}{=}\PY{n}{color}\PY{p}{,} \PY{n}{ls}\PY{o}{=}\PY{l+s+s1}{\PYZsq{}}\PY{l+s+s1}{:}\PY{l+s+s1}{\PYZsq{}}\PY{p}{)}
    
\PY{k}{for} \PY{n}{i\PYZus{}c}\PY{p}{,} \PY{n}{n\PYZus{}c} \PY{o+ow}{in} \PY{n+nb}{enumerate}\PY{p}{(}\PY{n}{bincount\PYZus{}conv}\PY{p}{)}\PY{p}{:}
    \PY{n}{plt}\PY{o}{.}\PY{n}{scatter}\PY{p}{(}\PY{l+m+mf}{0.4}\PY{p}{,} \PY{n}{i\PYZus{}c}\PY{p}{,} \PY{n}{color}\PY{o}{=}\PY{n}{color}\PY{p}{,} \PY{n}{s}\PY{o}{=}\PY{n}{n\PYZus{}c}\PY{o}{*}\PY{l+m+mi}{40}\PY{p}{)}
    \PY{n}{plt}\PY{o}{.}\PY{n}{text}\PY{p}{(}\PY{l+m+mf}{0.535}\PY{p}{,} \PY{n}{i\PYZus{}c}\PY{p}{,} \PY{n+nb}{str}\PY{p}{(}\PY{n}{n\PYZus{}c}\PY{p}{)}\PY{o}{+} \PY{l+s+s2}{\PYZdq{}}\PY{l+s+s2}{x}\PY{l+s+s2}{\PYZdq{}}\PY{p}{,} \PY{n}{ha}\PY{o}{=}\PY{l+s+s1}{\PYZsq{}}\PY{l+s+s1}{left}\PY{l+s+s1}{\PYZsq{}}\PY{p}{,} \PY{n}{va}\PY{o}{=}\PY{l+s+s1}{\PYZsq{}}\PY{l+s+s1}{center}\PY{l+s+s1}{\PYZsq{}}\PY{p}{)}

\PY{k}{for} \PY{n}{low\PYZus{}c}\PY{p}{,} \PY{n}{high\PYZus{}c} \PY{o+ow}{in} \PY{n+nb}{zip}\PY{p}{(}\PY{n}{low\PYZus{}dense\PYZus{}ls}\PY{p}{,} \PY{n}{high\PYZus{}dense\PYZus{}ls}\PY{p}{)}\PY{p}{:}
    \PY{n}{offset} \PY{o}{=} \PY{l+m+mi}{1} \PY{o}{+} \PY{n}{rng}\PY{o}{.}\PY{n}{randn}\PY{p}{(}\PY{l+m+mi}{1}\PY{p}{)} \PY{o}{*} \PY{l+m+mf}{0.1}
    \PY{n}{tried\PYZus{}cs} \PY{o}{=} \PY{n}{np}\PY{o}{.}\PY{n}{arange}\PY{p}{(}\PY{n}{low\PYZus{}c}\PY{p}{,} \PY{n}{high\PYZus{}c}\PY{o}{+}\PY{l+m+mi}{1}\PY{p}{)}
    \PY{n}{plt}\PY{o}{.}\PY{n}{plot}\PY{p}{(}\PY{p}{[}\PY{n}{offset}\PY{p}{,}\PY{p}{]} \PY{o}{*} \PY{n+nb}{len}\PY{p}{(}\PY{n}{tried\PYZus{}cs}\PY{p}{)}\PY{p}{,} \PY{n}{tried\PYZus{}cs}\PY{p}{,} \PY{n}{marker}\PY{o}{=}\PY{l+s+s1}{\PYZsq{}}\PY{l+s+s1}{o}\PY{l+s+s1}{\PYZsq{}}\PY{p}{,} \PY{n}{alpha}\PY{o}{=}\PY{l+m+mf}{0.5}\PY{p}{,} \PY{n}{color}\PY{o}{=}\PY{n}{color}\PY{p}{,} \PY{n}{ls}\PY{o}{=}\PY{l+s+s1}{\PYZsq{}}\PY{l+s+s1}{:}\PY{l+s+s1}{\PYZsq{}}\PY{p}{)}
    
\PY{k}{for} \PY{n}{i\PYZus{}c}\PY{p}{,} \PY{n}{n\PYZus{}c} \PY{o+ow}{in} \PY{n+nb}{enumerate}\PY{p}{(}\PY{n}{bincount\PYZus{}dense}\PY{p}{)}\PY{p}{:}
    \PY{n}{plt}\PY{o}{.}\PY{n}{scatter}\PY{p}{(}\PY{l+m+mf}{1.4}\PY{p}{,} \PY{n}{i\PYZus{}c}\PY{p}{,} \PY{n}{color}\PY{o}{=}\PY{n}{color}\PY{p}{,} \PY{n}{s}\PY{o}{=}\PY{n}{n\PYZus{}c}\PY{o}{*}\PY{l+m+mi}{40}\PY{p}{)}
    \PY{n}{plt}\PY{o}{.}\PY{n}{text}\PY{p}{(}\PY{l+m+mf}{1.535}\PY{p}{,} \PY{n}{i\PYZus{}c}\PY{p}{,} \PY{n+nb}{str}\PY{p}{(}\PY{n}{n\PYZus{}c}\PY{p}{)}\PY{o}{+} \PY{l+s+s2}{\PYZdq{}}\PY{l+s+s2}{x}\PY{l+s+s2}{\PYZdq{}}\PY{p}{,} \PY{n}{ha}\PY{o}{=}\PY{l+s+s1}{\PYZsq{}}\PY{l+s+s1}{left}\PY{l+s+s1}{\PYZsq{}}\PY{p}{,} \PY{n}{va}\PY{o}{=}\PY{l+s+s1}{\PYZsq{}}\PY{l+s+s1}{center}\PY{l+s+s1}{\PYZsq{}}\PY{p}{)}

\PY{n}{plt}\PY{o}{.}\PY{n}{xlim}\PY{p}{(}\PY{o}{\PYZhy{}}\PY{l+m+mf}{0.5}\PY{p}{,}\PY{l+m+mi}{2}\PY{p}{)}
\PY{n}{plt}\PY{o}{.}\PY{n}{xlabel}\PY{p}{(}\PY{l+s+s2}{\PYZdq{}}\PY{l+s+s2}{Type of layer}\PY{l+s+s2}{\PYZdq{}}\PY{p}{)}
\PY{n}{plt}\PY{o}{.}\PY{n}{ylabel}\PY{p}{(}\PY{l+s+s2}{\PYZdq{}}\PY{l+s+s2}{Number of layers}\PY{l+s+s2}{\PYZdq{}}\PY{p}{)}
\PY{n}{plt}\PY{o}{.}\PY{n}{xticks}\PY{p}{(}\PY{p}{[}\PY{l+m+mi}{0}\PY{p}{,}\PY{l+m+mi}{1}\PY{p}{]}\PY{p}{,} \PY{p}{[}\PY{l+s+s2}{\PYZdq{}}\PY{l+s+s2}{Convolutional}\PY{l+s+s2}{\PYZdq{}}\PY{p}{,} \PY{l+s+s2}{\PYZdq{}}\PY{l+s+s2}{Dense}\PY{l+s+s2}{\PYZdq{}}\PY{p}{]}\PY{p}{,} \PY{n}{rotation}\PY{o}{=}\PY{l+m+mi}{45}\PY{p}{)}
\PY{n}{plt}\PY{o}{.}\PY{n}{yticks}\PY{p}{(}\PY{p}{[}\PY{l+m+mi}{1}\PY{p}{,}\PY{l+m+mi}{2}\PY{p}{,}\PY{l+m+mi}{3}\PY{p}{,}\PY{l+m+mi}{4}\PY{p}{,}\PY{l+m+mi}{5}\PY{p}{,}\PY{l+m+mi}{6}\PY{p}{,}\PY{l+m+mi}{7}\PY{p}{]}\PY{p}{)}\PY{p}{;}
\PY{n}{plt}\PY{o}{.}\PY{n}{title}\PY{p}{(}\PY{l+s+s2}{\PYZdq{}}\PY{l+s+s2}{Number of layers in prior works}\PY{l+s+s2}{\PYZsq{}}\PY{l+s+s2}{ architectures}\PY{l+s+s2}{\PYZdq{}}\PY{p}{,} \PY{n}{y}\PY{o}{=}\PY{l+m+mf}{1.05}\PY{p}{)}
\PY{n}{glue}\PY{p}{(}\PY{l+s+s1}{\PYZsq{}}\PY{l+s+s1}{layernum\PYZus{}fig}\PY{l+s+s1}{\PYZsq{}}\PY{p}{,} \PY{n}{fig}\PY{p}{)}
\PY{n}{plt}\PY{o}{.}\PY{n}{close}\PY{p}{(}\PY{n}{fig}\PY{p}{)}
\PY{k+kc}{None}
\end{Verbatim}
\end{tcolorbox}

    \begin{center}
    \adjustimage{max size={0.9\linewidth}{0.9\paperheight}}{PriorWork_files/PriorWork_11_0.png}
    \end{center}
    { \hspace*{\fill} \\}
    
    ```\{glue:figure\} layernum\_fig

\textbf{Number of layers in prior work.} Small grey markers represent
individual architectures. Dashed lines indicate different number of
layers investigated in a single study (e.g., a single study investigated
3-7 convolutional layers). Larger grey markers indicate sum of
occurences of that layer number over all studies (e.g., 9 architectures
used 2 convolutional layers). Note most architectures use only 1-3
convolutional layers. ```

    The architectures used in prior work typically only included up to 3
layers, with only 2 studies considering more layers. As network
architectures in other domains tend to be a lot deeper, we also
evaluated architectures with a larger number of layers in our work.
Several architectures from prior work also included fully-connected
layers with larger number of parameters which had fallen out of favor in
computer-vision deep-learning architectures due to their large compute
and memory requirements with little accuracy benefit. Our architectures
do not include traditional fully-connected layers with a large number of
parameters.

    \hypertarget{hyperparameter-evaluations}{%
\section{Hyperparameter
Evaluations}\label{hyperparameter-evaluations}}

    ```\{table\} Design choices and training strategies that prior
deep-learning EEG decoding studies had studies. :name:
prior-work-design-choices-table

\begin{longtable}[]{@{}
  >{\raggedright\arraybackslash}p{(\columnwidth - 4\tabcolsep) * \real{0.1944}}
  >{\raggedright\arraybackslash}p{(\columnwidth - 4\tabcolsep) * \real{0.3611}}
  >{\raggedright\arraybackslash}p{(\columnwidth - 4\tabcolsep) * \real{0.4444}}@{}}
\toprule
\begin{minipage}[b]{\linewidth}\raggedright
Study
\end{minipage} & \begin{minipage}[b]{\linewidth}\raggedright
Design choices
\end{minipage} & \begin{minipage}[b]{\linewidth}\raggedright
Training strategies
\end{minipage} \\
\midrule
\endhead
\cite{lawhern\_eegnet:\_2016} & Kernel sizes & \\
\cite{sun\_remembered\_2016} & & Different time windows \\
\cite{tabar\_novel\_2017} & Addition of six-layer stacked
autoencoder on ConvNet features Kernel sizes & \\
\cite{liang\_predicting\_2016} & & Different subdivisions of
frequency range Different lengths of time crops Transfer learning with
auxiliary non-epilepsy datasets \\
\cite{hajinoroozi\_eeg-based\_2016} & Replacement of
convolutional layers by restricted Boltzmann machines with slightly
varied network architecture\} & \\
\cite{antoniades\_deep\_2016} & 1 or 2 convolutional layers
& \\
\cite{page\_wearable\_2016} & & Cross-subject supervised
training, within-subject finetuning of fully connected layers \\
\cite{bashivan\_learning\_2016} & Number of convolutional
layers Temporal processing of ConvNet output by max pooling, temporal
convolution, LSTM or temporal convolution + LSTM & \\
\cite{stober\_learning\_2016} & Kernel sizes & Pretraining
first layer as convolutional autoencoder with different constraints \\
\cite{sakhavi\_parallel\_2015} & Combination ConvNet and MLP
(trained on different features) vs.~only ConvNet vs.~only MLP & \\
\cite{stober\_using\_2014} & Best values from automatic
hyperparameter optimization: frequency cutoff, one vs two layers, kernel
sizes, number of channels, pooling width & Best values from automatic
hyperparameter optimization: learning rate, learning rate decay,
momentum, final momentum \\
\cite{wang\_deep\_2013} & Partially supervised CSA & \\
\cite{cecotti\_convolutional\_2011} & Electrode subset (fixed
or automatically determined) Using only one spatial filter Different
ensembling strategies & \\
\bottomrule
\end{longtable}

```

    Prior work varied widely in their comparison of design choices and
training strategies. 6 of the studies did not compare any design choices
or training strategy hyperparameters. The other 13 studies evaluated
different hyperparameters, with the most common one the kernel size (see
\Cref{prior-work-design-choices-table}). Only one study
evaluated a wider range of hyperparameters
\cite{stober\_using\_2014}. To fill this gap, we compared a
wider range of design choices and training strategies and specifically
evaluated whether improvements of computer vision architecture design
choices and training strategies also lead to improvements in EEG
decoding.

    \hypertarget{visualizations}{%
\section{Visualizations}\label{visualizations}}

    \texttt{\{table\}\ Visualizations\ presented\ in\ prior\ work.\ :name:\ prior-work-visualizations-table\ \textbar{}\ Study\ \textbar{}\ Visualization\ type(s)\ \textbar{}\ Visualization\ findings\ \ \ \textbar{}\ \textbar{}:-\/-\/-\/-\/-\/-\/-\textbar{}:-\/-\/-\/-\/-\textbar{}:-\/-\/-\/-\/-\textbar{}\ \textbar{}\{cite\}\textasciigrave{}sun\_remembered\_2016\textasciigrave{}\textbar{}\ Weights\ (spatial)\textbar{}\ Largest\ weights\ found\ over\ prefrontal\ and\ temporal\ cortex\textbar{}\ \textbar{}\{cite\}\textasciigrave{}manor\_multimodal\_2016\textasciigrave{}\ \textbar{}\ Weights\ \textless{}br\textgreater{}\ Activations\ \textless{}br\textgreater{}\ Saliency\ maps\ by\ gradient\ \textbar{}\ Weights\ showed\ typical\ P300\ distribution\ \textless{}br\textgreater{}Activations\ were\ high\ at\ plausible\ times\ (300-500ms)\ \textless{}br\textgreater{}Saliency\ maps\ showed\ plausible\ spatio-temporal\ plots\textbar{}\ \ \textbar{}\{cite\}
    \textasciigrave{}tabar\_novel\_2017\textasciigrave{}\ \textbar{}\ Weights\ (spatial\ +\ frequential)\ \textbar{}\ Some\ weights\ represented\ difference\ of\ values\ of\ two\ electrodes\ on\ different\ sides\ of\ head\textbar{}\ \textbar{}\{cite\}
    \textasciigrave{}liang\_predicting\_2016\textasciigrave{}\ \textbar{}\ Weights\ \textless{}br\textgreater{}\ Clustering\ of\ weights\ \textbar{}\ Clusters\ of\ weights\ showed\ typical\ frequency\ band\ subdivision\ (delta,\ theta,\ alpha,\ beta,\ gamma)\textbar{}\ \textbar{}\{cite\}
    \textasciigrave{}antoniades\_deep\_2016\textasciigrave{}\ \textbar{}\ Weights\ \textless{}br\textgreater{}Correlation\ weights\ and\ interictal\ epileptic\ discharges\ (IED)\ \textless{}br\textgreater{}Activations\ \ \ \ \ \ \ \ \ \ \ \ \ \ \ \ \ \ \ \ \ \ \ \ \textbar{}\ Weights\ increasingly\ correlated\ with\ IED\ waveforms\ with\ increasing\ number\ of\ training\ iterations\ \textless{}br\textgreater{}Second\ layer\ captured\ more\ complex\ and\ well-defined\ epileptic\ shapes\ than\ first\ layer\ \textless{}br\textgreater{}IEDs\ led\ to\ highly\ synchronized\ activations\ for\ neighbouring\ electrodes\ \textbar{}\ \textbar{}\{cite\}
    \textasciigrave{}thodoroff\_learning\_2016\textasciigrave{}\ \textbar{}\ Input\ occlusion\ and\ effect\ on\ prediction\ accuracy\ \textbar{}\ Allowed\ to\ locate\ areas\ critical\ for\ seizure\ \textbar{}\ \textbar{}\{cite\}\textasciigrave{}
    george\_single-trial\_2016\textasciigrave{}\ \textbar{}\ Weights\ (spatial)\ \textbar{}\ Some\ filter\ weights\ had\ expected\ topographic\ distributions\ for\ P300\ \textless{}br\textgreater{}Others\ filters\ had\ large\ weights\ on\ areas\ not\ traditionally\ associated\ with\ P300\textbar{}\ \textbar{}\{cite\}\textasciigrave{}bashivan\_learning\_2016\textasciigrave{}\ \textbar{}\ Inputs\ that\ maximally\ activate\ given\ filter\ \textless{}br\textgreater{}Activations\ of\ these\ inputs\ \textless{}br\textgreater{}"Deconvolution"\ for\ these\ inputs\ \textbar{}\ Different\ filters\ were\ sensitive\ to\ different\ frequency\ bands\ \textless{}br\textgreater{}Later\ layers\ had\ more\ spatially\ localized\ activations\ \textless{}br\textgreater{}Learned\ features\ had\ noticeable\ links\ to\ well-known\ electrophysiological\ markers\ of\ cognitive\ load\ \textless{}br\textgreater{}\ \textbar{}\ \textbar{}\{cite\}\textasciigrave{}stober\_learning\_2016\textasciigrave{}\ \textbar{}\ Weights\ (spatial+3\ timesteps,\ pretrained\ as\ autoencoder)\ \textbar{}\ Different\ constraints\ led\ to\ different\ weights,\ one\ type\ of\ constraints\ could\ enforce\ weights\ that\ are\ similar\ across\ subjects;\ other\ type\ of\ constraints\ led\ to\ weights\ that\ have\ similar\ spatial\ topographies\ under\ different\ architectural\ configurations\ and\ preprocessings\ \textbar{}\ \textbar{}\{cite\}\textasciigrave{}manor\_convolutional\_2015\textasciigrave{}\ \textbar{}\ Weights\ \textless{}br\textgreater{}\ Mean\ and\ single-trial\ activations\ \textbar{}\ Spatiotemporal\ regularization\ led\ to\ softer\ peaks\ in\ weights\ \textless{}br\textgreater{}Spatial\ weights\ showed\ typical\ P300\ distribution\ \textless{}br\textgreater{}Activations\ mostly\ had\ peaks\ at\ typical\ times\ (300-400ms)\ \textbar{}\ \textbar{}\{cite\}\textasciigrave{}cecotti\_convolutional\_2011\textasciigrave{}\ \textbar{}\ Weights\ \textbar{}\ Spatial\ filters\ were\ similar\ for\ different\ architectures\ \textless{}br\textgreater{}Spatial\ filters\ were\ different\ (more\ focal,\ more\ diffuse)\ for\ different\ subjects\ \textbar{}}

    Visualizations can help understand what information the networks are
extracting from the EEG signal. 11 of the prior 19 studies presented any
visualizations. These studies mostly focused on analyzing weights and
activations, see \Cref{prior-work-visualizations-table}. In
our work, we first focused on investigating how far the networks extract
spectral features known to work well for movement-related decoding, see
\Cref{perturbation-visualization}. Later, we also developed
more sophisticated visualization methods and applied them both to
pathology decoding, see \Cref{invertible-networks} and
\Cref{understanding-pathology}.

    \texttt{\{admonition\}\ Open\ Questions\ :class:\ tip\ *\ How\ do\ ConvNets\ perform\ on\ well-researched\ EEG\ movement-related\ decoding\ tasks\ against\ strong\ feature-based\ baselines?\ *\ How\ do\ shallower\ and\ deeper\ architectures\ compare?\ *\ How\ do\ design\ choices\ and\ training\ strategies\ affect\ the\ decoding\ performance?\ *\ What\ features\ do\ the\ deep\ networks\ learn\ on\ the\ EEG\ signals?\ *\ Do\ they\ learn\ to\ use\ higher-frequency\ (\textgreater{}50\ Hz)\ information?}


    % Add a bibliography block to the postdoc
    
    
    
\end{document}
