%************************************************
\chapter{Introduction}\label{ch:introduction}
%************************************************

\todotext{textbox}

Machine learning (ML), i.e., using data to learn programs that perform a
desired task, has the potential to benefit medical brain-signal-decoding
applications. Compared to humans, machine-learning programs can process
larger amounts of brain-signal data and may extract different
information. For example, machine-learning algorithms have been
developed to help doctors triage patients by quickly detecting stroke
biomarkers from computed tomography (CT)
\citep{chavva2022deep}, to enable brain-computer interfaces
by recognizing people's intentions from electroencephalographic (EEG) in
real time \citep{abiri2019comprehensive} and to detect
pathology from long brain signal recordings
\citep{gemein2020machine,schirrmeisterdeeppathology}. Also,
as brain signals are far from being fully understood, machine-learning
algorithms have the potential to advance scientific understanding by
finding new brain-signal biomarkers for different pathologies
\citep{raghu2020survey}.



    Electroencephalographic (EEG) brain-signal recordings are well-suited
for machine learning since they are easy to acquire, while being
time-consuming and challenging to manually interpret by doctors.
Generating large EEG datasets is relatively simple compared to other
medical recordings because of the low cost and minimal side effects of
performing EEG recordings. Furthermore, EEG recordings are particularly
challenging for humans to interpret, making them a promising target for
information extraction through machine learning. Some clinical
applications of EEG such as pathology diagnosis may be improved by
machine learning, while others such as brain-computer interfaces are
even only possible because of it. Finally, since the information
contained in the EEG signal is far from being fully understood, machine
learning may even help understand the EEG signal itself better.

Deep learning is a very promising approach for brain-signal decoding
from EEG. The term deep learning describes machine-learning models with
multiple computational stages, where the computational stages are
typically trained jointly to solve a given task
\citep{lecun_deep_2015,schmidhuber_deep_2015}.
Convolutional neural networks (ConvNets) are deep learning models that
are inspired by computational structures in the visual cortex of the
brain. ConvNets only have very
general assumptions about the properties of their training signals
embedded into them (such as smoothness and local-to-global hierarchical
structure) and have shown great success on a variety of  decoding tasks 
on natural signals, including object detection in images, speech recognition from audio or machine translation.
Therefore, ConvNets are very promising to apply to hard-to-understand
natural signals like EEG signals.

    Prior to the work presented in this thesis, it was unclear how well
ConvNet architectures can decode EEG signals compared to
hand-engineered, feature-based approaches. The high dimensionality, low
signal-to-noise ratio and large signal variability (e.g., from person to
person or even session to session for the same person) of EEG data
present challenges that may be better addressed by feature-based
approaches that exploit more specific assumptions about the EEG signal.
While there had been previous efforts to apply deep learning to EEG, a
systematic study of the performance of modern ConvNets on EEG decoding
compared with a strong feature-based baseline and including the impact
of network architecture and training hyperparameters, was lacking.
Furthermore, research into understanding what features the ConvNets
extract from the EEG signal had been limited.

    We therefore created several ConvNet architectures to thoroughly
evaluate on EEG decoding. We first evaluated our ConvNet's decoding
performance under a range of different hyperparameter choices on widely
studied movement-related decoding tasks like decoding which limb a
person is thinking of moving. On those tasks, we found our ConvNets to
perform at least as good as a strong feature-based baseline. The
ConvNets also generalized well to a range of other decoding tasks,
including other mental imageries, decoding whether a person made or
perceived an error, as well as pathology diagnosis.

We also developed visualizations to understand the features the ConvNets
extract from the EEG signal, finding that their predictions are
sensitive to plausible neurophysiological features. Using perturbations
of spectral features like amplitude and phase, we show topographies of
the causal effects of spectral changes on the networks predictions. For
decoding of executed movements, these topographies are consistent with
known movement-related spectral brain-signal changes like contralateral
alpha power decreases (e.g., decrease in alpha power on the right side
of the head when moving the left hand). They also suggest that networks
learn to use high-gamma information to predict the performed movement.
Visualizations of inputs that maximally activate specific units in one
of our ConvNets further reveals that the ConvNet has also learned
specific timecourses of amplitude changes, going beyond using just
averaged spectral features.

Later, we more deeply investigated features learned for pathology
decoding. Here, we made use of invertible networks, networks that are
designed such that you can invert intermediate and final network outputs
back to a corresponding input, allowing to visualize output changes in
input space. Further, we also developed a smaller network that is
specifically designed to be interpretable and train it to mimic the
invertible network. Using these methods we could directly show and
investigate temporal waveforms with spatial topographies that are
associated with pathological or healthy recordings. These visualizatiosn
revealed both neurophysiologically plausible features like temporal
slowing as a marker for a pathological EEG or occipital alpha as a
marker for healthy EEG, as well as surprising features like frontal and
temporal very-low-frequency 0.5 Hz components.

    In this thesis, I will first describe the research on deep learning EEG
decoding prior to our work, then proceed to describe the deep network
architectures and training methods we developed to rival or surpass
feature-based EEG decoding approaches on movement- and other
task-related EEG decoding tasks as well as pathology diagnosis.
Furthermore, I also describe the visualization methods we developed that
suggest the networks are using plausible neurophysiological patterns to
solve their tasks. In two separate method and result chapters, I will
delve more deeply into understanding the learned features for pathology
decoding, including by using invertible networks. Finally, I conclude
with my thoughts on the current state of EEG deep learning decoding and
promising avenues for further work like cross-dataset decoding models as
well as models that can process larger timescales of EEG signals.

\todotext{textbox}
